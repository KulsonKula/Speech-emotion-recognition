\documentclass[12pt,titlepage]{article}

\usepackage{geometry}
\geometry{
    a4paper,
    total={210mm,297mm},
    left=20mm,
    right=20mm,
    top=20mm,
    bottom=20mm,
}

	
\usepackage[]{subcaption}

\usepackage{pdflscape}

% Polski
\usepackage[]{polski} 
\usepackage[polish]{babel}

%do tabel
\usepackage{multirow}

% Pierwszy akapit - wcięty
\usepackage[]{indentfirst}

% Matematyka
\usepackage[]{amsfonts}

\usepackage[]{amsmath}

% Formatowanie
\usepackage{ragged2e}

% Tytuły sekcji
\usepackage{titlesec}
%\titleformat{\section}[block]{\Large\bfseries}{}{1em}{}

% <=
\usepackage{amssymb}

% eps
\usepackage{graphicx}
% \usepackage{subfigure}

% Tabele
\usepackage{array}

\usepackage[style=czech]{csquotes}

\renewcommand*{\thesubsubsection}{}

\usepackage{hyperref}
\hypersetup{
    colorlinks,
    citecolor=black,
    filecolor=black,
    linkcolor=black,
    urlcolor=black
}

\usepackage[numbered]{matlab-prettifier}
\lstset{
    literate={ą}{{\k{a}}}1
    {Ą}{{\k{A}}}1
    {ę}{{\k{e}}}1
    {Ę}{{\k{E}}}1
    {ó}{{\'o}}1
    {Ó}{{\'O}}1
    {ś}{{\'s}}1
    {Ś}{{\'S}}1
    {ł}{{\l{}}}1
    {Ł}{{\L{}}}1
    {ż}{{\.z}}1
    {Ż}{{\.Z}}1
    {ź}{{\'z}}1
    {Ź}{{\'Z}}1
    {ć}{{\'c}}1
    {Ć}{{\'C}}1
    {ń}{{\'n}}1
    {Ń}{{\'N}}1
}

\title{
\includegraphics[scale=0.75]{img/politechnika_sl_logo_bw_poziom_pl.eps}\\
\textbf{Wydział Automatyki, Elektroniki\\
i Informatyki}\\
\vspace*{1cm}
Systemy Interaktywne i Multimedialne \\ Projekt \\ Detekcja emocji w głosie

\vspace*{5cm}
}
\author{
Natalia Stręk,
Jakub Kula,
Paweł Wójtowicz
} 
\date{Gliwice 2023}

\begin{document}

\maketitle

\tableofcontents

\newpage


\section{Wizja projektu}
Projekt ten ma na celu stworzenie systemu zdolnego do detekcji emocji w ludzkim głosie takich jak radość, smutek, złość czy strach. Człowiek przede wszystkim rozpoznaje emocje za pomocą komunikacji niewerbalnej. Dobra sztuczna inteligencja powinna więc być zdolna do lepszego rozpoznawania emocji w głosie człowieka, wykorzystując do tego nieoczywiste i niezauważalne wzorce, które są trudne do dostrzeżenia na pierwszy rzut oka. System będzie zrealizowany przy wykorzystaniu technologi do przetwarzania sygnałów dźwiękowych jak i sztucznej inteligencji. Realizacja tego projektu może przynieść korzyści w usprawnieniu interakcji człowiek-maszyna dzięki czemu maszyna będzie mogła lepiej zrozumieć intencje oraz potrzeby użytkownika. Inną korzyścią płynącą z tego projektu może być badanie nad dynamiką emocji w mediach społecznościowych i transmisiach na żywo co pozwoli na lepsze zrozumienie reakcji społecznych. Wraz z rozwojem projektu, możliwe jest również zastosowanie detekcji emocji w ludzkim głosie obszarach takich jak medycyna. Takie systemy mogą być szczególnie pomocne w diagnozowaniu i leczeniu zaburzeń zdrowia psychnicznego.

\section{Cele projektu}
\begin{itemize}
    \item Opracowanie modelu uczenia maszynowego, który będzie potrafił rozpoznawać emocje w nagraniach głosowych.
          % \item Porównanie stworzonego modelu z istniejącymi już modelami do rozpoznawania emocji w głosie
    \item Stworzenie interaktywnego narzędzia umożliwiającego przesłanie nagrania głosowego i otrzymanie informacji zwrotnej na temat emocji wykrytych w głosie.
\end{itemize}

\section{Zakres projektu}
Projekt obejmuje stworzenie systemu zdolnego do detekcji szerokiego zakresu emocji w ludzkim głosie, włączając w to radość, smutek, złość, zaskoczenie, obojętność, spokój, odraza i strach. Wykorzystując technologie do przetwarzania sygnałów dźwiękowych oraz sztucznej inteligencji, projekt będzie skoncentrowany na identyfikację wzorców w ludzkim głosie, które mogą wskazywać na określoną emocję.

W ramach projektu zostanie opracowany i przetestowany model uczenia maszynowego o określonej dokładności w rozpoznawaniu emocji w ludzkim głosie, stworzenie interaktywnego interfejsu użytkownika pozwalającego na przesłanie nagrania głosowego i otrzymanie odpowiedzi na tematy wykrytych emocji.

Projekt skupia się wyłącznie na analizie emocji w ludzkim głosie i nie obejmuje dodatkowych analiz, takich jak funkcje przetwarzania języka naturalnego w celu lepszego zrozumienia kontekstu emocjonalnego, ani rozpoznawanie emocji w innych formach komunikacji niewerbalnej, takich jak mimika twarzy czy wykonywane gesty. Dodatkowo, projekt skoncentrowany jest wyłącznie na identyfikacji emocji w nagraniach głosowych i nie będzie obejmować analizy innych źródeł danych.

% Zakres definiuje granice projektu przedstawiając co zostanie wykonane w ramach projektu, a co nie. To co zostanie wykonane powinno zostać ujęte za pomocą konkretnych, mierzalnych rezultatów. (ok 200 słów)\\

\section{Etapy w projekcie}
\begin{itemize}
    \item Zgromadzenie i przetworzenie danych - dane w postaci krótkich nagrań dźwiękowych reprezentujące różne emocje.
    \item Wybór oraz ekstrakcja cech.
    \item Wybór architektury sieci, hiperparametrów, podział danych na zbiór uczący oraz treningowy.
    \item Trenowanie modelu i optymalizacja parametrów modelu.
    \item Ocena i walidacja modelu.
    \item Dostosowywanie parametrów modelu w celu poprawy jego wydajności.
    \item Stworzenie UI pozwalajacego użytkownikowi dodanie pliku dzwiękowego oraz nagrania własnej wiadomości przy pomocy wbudowanego mikrofonu
    \item Testowanie projektu
\end{itemize}

\section{Charakterystykę narzędzi}
\begin{itemize}
    \item Python - główny jezyk programowania. W nim zostaną napisane skrypty przetwarzajace dane oraz zostanie stworzony i nauczony model sieci neuronowe.
    \item Tensorflow i keras - Tenforflow będąca najpopularniejszą bibloteką uczenia maszynowego stanowi główne narzędzie do tworzenia sieci neuronowek. Keras udostępnia API do tworzenia modeli uczenia maszynowego.
    \item Numpy i Pandas - fundamentalne bibloteki do obliczeń naukowych w pythonie, umozliwiające efektyne przetwarzania dużych ilości danych oraz ich wnikliwą analize
    \item Librosa - bibloteka pozwalająca w łatwy sposób otworzyć i przetworzyć dane audio.
    \item Sckit-lern, Eli5 - bibloteki zawierające miedzy innymi bogaty zbiór algrytmów uczenia maszynowego. Pozwalające na uproszcznie etapu tworzenia sieci.
    \item Conda - Conda jest środowiskiem wirtualnym i systemem do zarządzania pakietami. Pozwala ona na tworzenie odseparowanych środowisk dla różnych projektów.
    \item Tkinter - bibloteka pozwalająca nam uprościc proces tworzenia działajacego interfejsu graficznego.
\end{itemize}
\end{document}