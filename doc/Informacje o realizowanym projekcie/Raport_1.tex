\documentclass[12pt,titlepage]{article}

\usepackage{geometry}
\geometry{
    a4paper,
    total={210mm,297mm},
    left=20mm,
    right=20mm,
    top=20mm,
    bottom=20mm,
}

	
\usepackage[]{subcaption}

\usepackage{pdflscape}

% Polski
\usepackage[]{polski} 
\usepackage[polish]{babel}

%do tabel
\usepackage{multirow}

% Pierwszy akapit - wcięty
\usepackage[]{indentfirst}

% Matematyka
\usepackage[]{amsfonts}

\usepackage[]{amsmath}

% Formatowanie
\usepackage{ragged2e}

% Tytuły sekcji
\usepackage{titlesec}
%\titleformat{\section}[block]{\Large\bfseries}{}{1em}{}

% <=
\usepackage{amssymb}

% eps
\usepackage{graphicx}
% \usepackage{subfigure}

% Tabele
\usepackage{array}

\usepackage[style=czech]{csquotes}

\renewcommand*{\thesubsubsection}{}

\usepackage{hyperref}
\hypersetup{
    colorlinks,
    citecolor=black,
    filecolor=black,
    linkcolor=black,
    urlcolor=black
}

\usepackage[numbered]{matlab-prettifier}
\lstset{
    literate={ą}{{\k{a}}}1
    {Ą}{{\k{A}}}1
    {ę}{{\k{e}}}1
    {Ę}{{\k{E}}}1
    {ó}{{\'o}}1
    {Ó}{{\'O}}1
    {ś}{{\'s}}1
    {Ś}{{\'S}}1
    {ł}{{\l{}}}1
    {Ł}{{\L{}}}1
    {ż}{{\.z}}1
    {Ż}{{\.Z}}1
    {ź}{{\'z}}1
    {Ź}{{\'Z}}1
    {ć}{{\'c}}1
    {Ć}{{\'C}}1
    {ń}{{\'n}}1
    {Ń}{{\'N}}1
}

\title{
\includegraphics[scale=0.75]{img/politechnika_sl_logo_bw_poziom_pl.eps}\\
\textbf{Wydział Automatyki, Elektroniki\\
i Informatyki}\\
\vspace*{1cm}
Systemy Interaktywne i Multimedialne \\ Projekt \\ Detekcja emocji w głosie

\vspace*{5cm}
}
\author{
Natalia Stręk,
Jakub Kula,
Paweł Wójtowicz
} 
\date{Gliwice 2023}

\begin{document}

\maketitle

\tableofcontents

\newpage


\section{Wizja projektu}
DO TO NAPISAĆ TO!
\begin{itemize}
    \item prexyxyjna detekcja acc>=60\%
    \item efektyność w czasie rzeczywistym
\end{itemize}

\section{Cele projektu}
\begin{itemize}
    \item Opracowanie modelu uczenia maszynowego, który będzie potrafił rozpoznawać emocje w nagraniach głosowych.
    \item Stworzenie interaktywnego narzędzia umożliwiającego przesłanie nagrania głosowego i otrzymanie informacji zwrotnej na temat emocji wykrytych w głosie.
\end{itemize}

\section{Zakres projektu}
DO TO NAPISAĆ TO!
Zakres projektu będzie obejomwać wybór odpowiednich danych. Dane muszą zawierać odpowiednią ilość nagrań oraz obejmowac odpowiednią ilość emocji.
Następnym krokiem będzie wybór odpowiednich cech które posłużą do nauczenia modelu. Po wyborze odpowiednich cech zostanie sporządzony skrypt przetwarzające wybrane dane. Kolejnym etapem będą badania nad achiterkórą i zestawem hisperparamterów które zapewnią najwyższy wspólcznnik dokłądności. Następnie zostanie wybrana o

Zakres definiuje granice projektu przedstawiając co zostanie wykonane w ramach projektu, a co nie. To co zostanie wykonane powinno zostać ujęte za pomocą konkretnych, mierzalnych rezultatów. (ok 200 słów)\\

\section{Etapy w projekcie}
\begin{itemize}
    \item Znalezienie nagrań głosowych przedstawiające różne emocje.
    \item Reserach na temat sygnałów dzwiękowych i ich cech
    \item Wybór cech które posłużą do nauki modelu uczenia maszynowego
    \item Przetworzenie nagrań głosowych, wyciąganiecie z nich cech charakterystyczych
    \item wybór metody kroswalidacji
    \item Strojenie i wybór architektóry sieci oraz jej hisperparamterów przy użyciu wyników z kroswalidacji
    \item Stworzenie UI pozwalajacego użytkownikowi dodanie pliku dzwiękowego oraz nagrania własnej wiadomości przy pomocy wbudowanego mikrofonu
    \item Testowanie projektu
\end{itemize}

\section{Charakterystykę narzędzi}
\begin{itemize}
    \item Python - główny jezyk programowania. W nim zostaną napisane skrypty przetwarzajace dane oraz zostanie stworzony i nauczony model sieci neuronowe.
    \item Tensorflow i keras - Tenforflow będąca najpopularniejszą bibloteką uczenia maszynowego stanowi główne narzędzie do tworzenia sieci neuronowek. Keras udostępnia API do tworzenia modeli uczenia maszynowego.
    \item Numpy i Pandas - fundamentalne bibloteki do obliczeń naukowych w pythonie, umozliwiające efektyne przetwarzania dużych ilości danych oraz ich wnikliwą analize
    \item Librosa - bibloteka pozwalająca w łatwy sposób otworzyć i przetworzyć dane audio.
    \item Sckit-lern, Eli5 - bibloteki zawierające miedzy innymi bogaty zbiór algrytmów uczenia maszynowego. Pozwalające na uproszcznie etapu tworzenia sieci.
    \item Conda - Conda jest środowiskiem wirtualnym i systemem do zarządzania pakietami. Pozwala ona na tworzenie odseparowanych środowisk dla różnych projektów.
    \item Tkinter - bibloteka pozwalająca nam uprościc proces tworzenia działajacego interfejsu graficznego.
\end{itemize}
\end{document}