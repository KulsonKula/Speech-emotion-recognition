\documentclass[12pt,titlepage]{article}

\usepackage{geometry}
\geometry{
    a4paper,
    total={210mm,297mm},
    left=20mm,
    right=20mm,
    top=20mm,
    bottom=20mm,
}

	
\usepackage[]{subcaption}

\usepackage{pdflscape}

% Polski
\usepackage[]{polski} 
\usepackage[polish]{babel}

%do tabel
\usepackage{multirow}

% Pierwszy akapit - wcięty
\usepackage[]{indentfirst}

% Matematyka
\usepackage[]{amsfonts}

\usepackage[]{amsmath}

% Formatowanie
\usepackage{ragged2e}

% Tytuły sekcji
\usepackage{titlesec}
%\titleformat{\section}[block]{\Large\bfseries}{}{1em}{}

% <=
\usepackage{amssymb}

% eps
\usepackage{graphicx}
% \usepackage{subfigure}

% Tabele
\usepackage{array}

\usepackage[style=czech]{csquotes}

\renewcommand*{\thesubsubsection}{}

\usepackage{hyperref}
\hypersetup{
    colorlinks,
    citecolor=black,
    filecolor=black,
    linkcolor=black,
    urlcolor=black
}

\usepackage[numbered]{matlab-prettifier}
\lstset{
    literate={ą}{{\k{a}}}1
    {Ą}{{\k{A}}}1
    {ę}{{\k{e}}}1
    {Ę}{{\k{E}}}1
    {ó}{{\'o}}1
    {Ó}{{\'O}}1
    {ś}{{\'s}}1
    {Ś}{{\'S}}1
    {ł}{{\l{}}}1
    {Ł}{{\L{}}}1
    {ż}{{\.z}}1
    {Ż}{{\.Z}}1
    {ź}{{\'z}}1
    {Ź}{{\'Z}}1
    {ć}{{\'c}}1
    {Ć}{{\'C}}1
    {ń}{{\'n}}1
    {Ń}{{\'N}}1
}

\title{
\includegraphics[scale=0.75]{img/politechnika_sl_logo_bw_poziom_pl.eps}\\
\textbf{Wydział Automatyki, Elektroniki\\
i Informatyki}\\
\vspace*{1cm}
Systemy Interaktywne i Multimedialne \\ Projekt \\ Detekcja emocji w głosie

\vspace*{5cm}
}
\author{
Natalia Stręk
Jakub Kula,
Paweł Wójtowicz,
} 
\date{Gliwice 2023}

\begin{document}

\maketitle

\tableofcontents

\newpage
\section{Analiza wyników i wnioski}
Przedstawić najważniejsze scenariusze użycia stworzonego oprogramowania(zazwyczaj od 3 do 6 scenariuszy). Scenariusze zilustrować zrzutami ekranowymi, do których referencje mają znaleźć się w tekście.\\

Do sprawozdania można dołączyć wideo prezentujące użycie scenariuszy (w osobnym pliku sprawozdania). Także w tym wypadku zespół, może być nadal zobowiązany do prezentacji wyników projektu na zajęciach.\\

Czy projekt zrealizowany został zgodnie z założeniami? Jeśli nie,podać przyczyny odstępstw od pierwotnych założeń. Należy odnieść się do zaakceptowanego dokumentu Informacje o realizowanym projekcie (ok 100 słów)\\

Czy projekt zrealizowany został zgodnie z harmonogramem? Jeśli nie, podać przyczyny opóźnień w odniesieniu do etapów pracy oraz opisać działania naprawcze podjęte aby osiągnąć wyznaczone cele projektu.\\

Czy wystąpiły jakieś nadzwyczajne wydarzenia w trakcie realizacji projektu? Czy wprowadzono jakieś znaczące zmiany do zrealizowanych działań lub w składzie zespołu? Jeśli tak, opisać jakie oraz podać ich wpływ na realizację projektu.\\

Doświadczenia uzyskane podczas realizacji projektu. Opisać jaką nową wiedzę uzyskał uczestnik projektu w czasie jego realizacji np. nowe metody programowania, nowe biblioteki, nowe narzędzie itp. (ok 200 słów)\\

Proponowane ulepszenia projektu. Opisać możliwe zmiany w systemie, użytej technologii, narzędziach, które mogłyby prowadzić do ulepszenia projektu.\\

Dodać skompresowany kod źródłowy (w osobnym pliku sprawozdania)
\end{document}